\documentclass[9pt,english]{extarticle}\usepackage[]{graphicx}\usepackage[]{color}
%% maxwidth is the original width if it is less than linewidth
%% otherwise use linewidth (to make sure the graphics do not exceed the margin)
\makeatletter
\def\maxwidth{ %
  \ifdim\Gin@nat@width>\linewidth
    \linewidth
  \else
    \Gin@nat@width
  \fi
}
\makeatother

\definecolor{fgcolor}{rgb}{0.345, 0.345, 0.345}
\newcommand{\hlnum}[1]{\textcolor[rgb]{0.686,0.059,0.569}{#1}}%
\newcommand{\hlstr}[1]{\textcolor[rgb]{0.192,0.494,0.8}{#1}}%
\newcommand{\hlcom}[1]{\textcolor[rgb]{0.678,0.584,0.686}{\textit{#1}}}%
\newcommand{\hlopt}[1]{\textcolor[rgb]{0,0,0}{#1}}%
\newcommand{\hlstd}[1]{\textcolor[rgb]{0.345,0.345,0.345}{#1}}%
\newcommand{\hlkwa}[1]{\textcolor[rgb]{0.161,0.373,0.58}{\textbf{#1}}}%
\newcommand{\hlkwb}[1]{\textcolor[rgb]{0.69,0.353,0.396}{#1}}%
\newcommand{\hlkwc}[1]{\textcolor[rgb]{0.333,0.667,0.333}{#1}}%
\newcommand{\hlkwd}[1]{\textcolor[rgb]{0.737,0.353,0.396}{\textbf{#1}}}%

\usepackage{framed}
\makeatletter
\newenvironment{kframe}{%
 \def\at@end@of@kframe{}%
 \ifinner\ifhmode%
  \def\at@end@of@kframe{\end{minipage}}%
  \begin{minipage}{\columnwidth}%
 \fi\fi%
 \def\FrameCommand##1{\hskip\@totalleftmargin \hskip-\fboxsep
 \colorbox{shadecolor}{##1}\hskip-\fboxsep
     % There is no \\@totalrightmargin, so:
     \hskip-\linewidth \hskip-\@totalleftmargin \hskip\columnwidth}%
 \MakeFramed {\advance\hsize-\width
   \@totalleftmargin\z@ \linewidth\hsize
   \@setminipage}}%
 {\par\unskip\endMakeFramed%
 \at@end@of@kframe}
\makeatother

\definecolor{shadecolor}{rgb}{.97, .97, .97}
\definecolor{messagecolor}{rgb}{0, 0, 0}
\definecolor{warningcolor}{rgb}{1, 0, 1}
\definecolor{errorcolor}{rgb}{1, 0, 0}
\newenvironment{knitrout}{}{} % an empty environment to be redefined in TeX

\usepackage{alltt}
%% definition of commands
\newcommand{\myTitle}{Research of variants in brain diseases\xspace}
\newcommand{\myName}{Sleiman Bassim\xspace}
\newcommand{\myUni}{University of Vermont\xspace}
\newcommand{\myFaculty}{Department of Microbiology, Molecular Genetics, and Computer Science\xspace}
\newcommand{\mySubtitle}{Postdoc Research\xspace}
\newcommand{\myDegree}{PhD\xspace}
\newcommand{\myLocation}{Vermont\xspace}
\newcommand{\myTime}{2014-2015\xspace}

%% get the spacing after macros right
\usepackage{xspace}

%% set margins for note insertion
\usepackage{marginnote}
\usepackage{todonotes}

%% margins and margin notes
\usepackage[top=1.5cm, bottom=1.5cm, left=.5cm, right=.5cm, outer=5cm, inner=2cm, marginparwidth=4cm, marginparsep=.35cm]{geometry}

%% line numbering
\usepackage{lineno}

%% fonts
%\usepackage{lmodern}
\usepackage[sc]{mathpazo}
\usepackage[T1]{fontenc} 
\usepackage[utf8]{inputenc}

%% language
\usepackage{babel}

%% math
\usepackage{amssymb,amsmath,wasysym}

%% floats
\usepackage{float,fixltx2e}

%% references bibliography
\usepackage[round,colon]{natbib}

%% formatter le titre des tables
\usepackage[font=small, labelfont=bf,belowskip=10pt,aboveskip=0pt]{caption, ragged2e}

%% remove space around inserted floating objects\\
\setlength{\intextsep}{10pt plus 2pt minus 2pt}	

%% line spacing
\linespread{1} % a bit more for Palatino

%% graphics and eps importation
\PassOptionsToPackage{pdftex}{graphicx}
\usepackage{graphicx,epstopdf,tocloft}

%% espace entre titre de section
\usepackage[compact]{titlesec} \titlespacing{\section}{0pt}{*0}{*0}
\titlespacing{\subsection}{0pt}{*0}{*0}
\titlespacing{\subsubsection}{0pt}{*0}{*0}

%% hyperlinks
\usepackage{hyperref} \hypersetup{
  % Uncomment the line below to remove all links
  colorlinks=true, linktocpage=true, pdfstartview=FitV,
  pdfstartpage=2,
  % Uncomment the line below if you want to have black links
  % colorlinks=false, linktocpage=false, pdfborder={0 0 0},
  % pdfstartpage=3, pdfstartview=FitV,
  breaklinks=true, pdfpagemode=UseNone, pageanchor=true,
  pdfpagemode=UseOutlines, plainpages=false, bookmarksnumbered,
  bookmarksopen=true, bookmarksopenlevel=1, hypertexnames=true,
  pdfhighlight=/O, urlcolor=red, linkcolor=RoyalBlue,
  citecolor=MidnightBlue,
  % PDF file meta-information
  pdftitle={\myTitle}, pdfauthor={\textcopyright\ \myName, \myUni,
    \myFaculty}, pdfsubject={Molecular Genetics}, pdfkeywords={Postdoc
    Research}, pdfcreator={pdfLaTeX}, pdfproducer={LaTeX with
    hyperref} }

%% paragraph indent and break
%% disable the standard Anglo-American publishers' practice of no indentation
\setlength{\parindent}{0cm}
\usepackage{indentfirst}


\title{R implementation}
\author{Sleiman Bassim, PhD}
\IfFileExists{upquote.sty}{\usepackage{upquote}}{}
\begin{document}
\maketitle
\begin{linenumbers}


 

\section{Explore Twitter using R}
\label{sec:explore}


\noindent
Load packages for data mining of tweets. \emph{twitteR} for accessing twitter, \emph{tm} for semantic mining, and \emph{wordcloud} to visualize the results.
\begin{knitrout}
\definecolor{shadecolor}{rgb}{0.969, 0.969, 0.969}\color{fgcolor}\begin{kframe}
\begin{alltt}
\hlkwd{library}\hlstd{(ROAuth);}
\hlkwd{library}\hlstd{(dplyr);}
\hlkwd{setwd}\hlstd{(}\hlstr{"/media/Data/Dropbox/R/twitter"}\hlstd{)}
\end{alltt}
\end{kframe}
\end{knitrout}

\subsection{Update to latest version of twitteR}
\label{subsec:twitteR}

\noindent
Install an updated version of \emph{twitteR} from github.
\marginnote{Installation from CRAN did not result in an updated version of \emph{twitteR}. Github installs a dev version.}[1cm]
Run all this chunk ONCE! then RESTART R session!
\begin{knitrout}
\definecolor{shadecolor}{rgb}{0.969, 0.969, 0.969}\color{fgcolor}\begin{kframe}
\begin{alltt}
\hlcom{#install.packages(c("devtools","rjson","bit64","httr"));}
\hlcom{#library(devtools);}
\hlcom{#install_github("twitteR",username="geoffjentry");}
\hlkwd{library}\hlstd{(twitteR);}
\end{alltt}
\end{kframe}
\end{knitrout}


\subsection{Authentication settings from Twitter}
\label{subsec:authenticate}

\noindent
From the application settings \href{https://apps.twitter.com/app/6985014/keys}{twitter} I will get the access keys to authenticate with twitter. Follow \href{http://geoffjentry.hexdump.org/twitteR.pdf}{this} tutorial under section 3 for additional settings.
Get keys after setting an app in Twitter \textbf{hint} for a valid URL use http://test.de/ 
\begin{knitrout}
\definecolor{shadecolor}{rgb}{0.969, 0.969, 0.969}\color{fgcolor}\begin{kframe}
\begin{alltt}
\hlkwd{source}\hlstd{(}\hlstr{"keys.R"}\hlstd{)}
\hlkwd{setup_twitter_oauth}\hlstd{(api_key, api_secret, access_token, access_secret)}
\end{alltt}
\begin{verbatim}
[1] "Using direct authentication"
\end{verbatim}
\end{kframe}
\end{knitrout}

\subsection{Testing authentication}
\label{subsec:test_oauth}

\noindent
Getting some of the trends that are popular. But first convert to dplyr wrapper dataframe.
\begin{knitrout}
\definecolor{shadecolor}{rgb}{0.969, 0.969, 0.969}\color{fgcolor}\begin{kframe}
\begin{alltt}
\hlstd{current_trends} \hlkwb{<-} \hlkwd{availableTrendLocations}\hlstd{()}
\hlkwd{head}\hlstd{(current_trends)}
\end{alltt}
\begin{verbatim}
       name country woeid
1 Worldwide             1
2  Winnipeg  Canada  2972
3    Ottawa  Canada  3369
4    Quebec  Canada  3444
5  Montreal  Canada  3534
6   Toronto  Canada  4118
\end{verbatim}
\begin{alltt}
\hlstd{montreal_trends} \hlkwb{<-} \hlkwd{getTrends}\hlstd{(}\hlnum{3534}\hlstd{)}
\hlstd{mt_df} \hlkwb{<-} \hlkwd{tbl_df}\hlstd{(montreal_trends)}
\hlstd{mt_df}
\end{alltt}
\begin{verbatim}
Source: local data frame [10 x 4]

              name
1           Québec
2         Montréal
3         #Salvail
4       Windows 10
5           Canada
6          Toronto
7            Texas
8      #1DProposal
9  #VerifiedNordan
10       Halloween
Variables not shown: url (chr), query (chr),
  woeid (chr)
\end{verbatim}
\end{kframe}
\end{knitrout}

\subsection{Transforming data }
\label{subsec:transformation}

\noindent
Retrieve a user current timeline
\begin{knitrout}
\definecolor{shadecolor}{rgb}{0.969, 0.969, 0.969}\color{fgcolor}\begin{kframe}
\begin{alltt}
\hlstd{keywords} \hlkwb{<-} \hlkwd{userTimeline}\hlstd{(}\hlstr{"genomicsedu"}\hlstd{,} \hlkwc{n}\hlstd{=}\hlnum{100}\hlstd{)}
\end{alltt}
\end{kframe}
\end{knitrout}


\noindent
The list of tweets is transformed into a data frame. You can use \verb|tbl_df| to wrapp the dataframe. 
Manipulate the text by removing generic and non relevant strings and clarifying the sentences and to keep important keywords from the tweets.
\verb|Do.call| works fine in R but not in knitr, so run the chunk in R first, then in knitr, then load the text mining (tm) package. 
\begin{knitrout}
\definecolor{shadecolor}{rgb}{0.969, 0.969, 0.969}\color{fgcolor}\begin{kframe}
\begin{alltt}
\hlkwd{require}\hlstd{(plyr);}
\hlstd{ask.api} \hlkwb{<-} \hlkwa{function}\hlstd{(}\hlkwc{hash}\hlstd{,}\hlkwc{n}\hlstd{)\{}
    \hlstd{a} \hlkwb{<-} \hlkwd{searchTwitter}\hlstd{(hash,} \hlkwc{n}\hlstd{=n)}
    \hlstd{b} \hlkwb{<-} \hlkwd{tbl_df}\hlstd{(}\hlkwd{do.call}\hlstd{(}\hlstr{"rbind.fill"}\hlstd{,}\hlkwd{lapply}\hlstd{(a, as.data.frame)))}
\hlstd{\}}
\hlstd{data.tweets} \hlkwb{<-} \hlkwd{ask.api}\hlstd{(}\hlkwc{hash}\hlstd{=}\hlstr{"#genomics"}\hlstd{,}\hlkwc{n}\hlstd{=}\hlnum{1000}\hlstd{)}
\end{alltt}


{\ttfamily\noindent\color{warningcolor}{Warning: 1000 tweets were requested but the API can only return 100}}

{\ttfamily\noindent\bfseries\color{errorcolor}{Error: cannot coerce class "{}structure("{}status"{}, package = "{}twitteR"{})"{} to a data.frame}}\begin{alltt}
\hlkwd{dim}\hlstd{(data.tweets)}
\end{alltt}
\begin{verbatim}
[1] 100  16
\end{verbatim}
\end{kframe}
\end{knitrout}

Build a text mining wrapper function then inspect the extracted dataframe.
\begin{knitrout}
\definecolor{shadecolor}{rgb}{0.969, 0.969, 0.969}\color{fgcolor}\begin{kframe}
\begin{alltt}
\hlkwd{library}\hlstd{(tm);}
\hlstd{rm.generic} \hlkwb{<-} \hlkwa{function}\hlstd{(}\hlkwc{x}\hlstd{,}\hlkwc{language}\hlstd{,}\hlkwc{words}\hlstd{)\{}
    \hlstd{a} \hlkwb{<-} \hlkwd{Corpus}\hlstd{(}\hlkwd{VectorSource}\hlstd{(x[,}\hlnum{1}\hlstd{]))}
    \hlstd{a} \hlkwb{<-} \hlkwd{tm_map}\hlstd{(a, removePunctuation)}
    \hlstd{a} \hlkwb{<-} \hlkwd{tm_map}\hlstd{(a, removeNumbers)}
    \hlstd{a} \hlkwb{<-} \hlkwd{tm_map}\hlstd{(a, tolower)}
    \hlstd{b} \hlkwb{<-} \hlkwd{c}\hlstd{(}\hlkwd{stopwords}\hlstd{(language),words)}
\hlcom{#    a <- tm_map(a, removeWords, b)}
    \hlstd{a} \hlkwb{<-} \hlkwd{tm_map}\hlstd{(a,PlainTextDocument)}
\hlstd{\}}
\hlstd{testing} \hlkwb{<-} \hlkwd{rm.generic}\hlstd{(data.tweets,} \hlkwc{lang}\hlstd{=}\hlstr{'english'}\hlstd{,} \hlkwc{words}\hlstd{=}\hlkwd{c}\hlstd{(}\hlstr{"the"}\hlstd{,}\hlstr{"tvtag"}\hlstd{,}\hlstr{"but"}\hlstd{))}
\hlcom{#inspect(testing)}
\end{alltt}
\end{kframe}
\end{knitrout}


\subsection{Build a document-term matrix to setup a cloud-word}

\label{subsec:label}
\label{subsec:wordcloud}

\noindent
Additional text mining can be used to find the origin of the used words and meging them together. It is called stemming. Finding the source of every word.
First, build the document-term matrix then inspects it and the most popular words to find associated terms and their score to the related hashtag (up).
\begin{knitrout}
\definecolor{shadecolor}{rgb}{0.969, 0.969, 0.969}\color{fgcolor}\begin{kframe}
\begin{alltt}
\hlstd{my.dt} \hlkwb{<-} \hlkwd{DocumentTermMatrix}\hlstd{(testing)}
\hlstd{my.dt}
\end{alltt}
\begin{verbatim}
<<DocumentTermMatrix (documents: 100, terms: 370)>>
Non-/sparse entries: 1030/35970
Sparsity           : 97%
Maximal term length: 24
Weighting          : term frequency (tf)
\end{verbatim}
\begin{alltt}
\hlkwd{findFreqTerms}\hlstd{(my.dt,} \hlkwc{lowfreq} \hlstd{=} \hlnum{30}\hlstd{)}
\end{alltt}
\begin{verbatim}
[1] "salvail"
\end{verbatim}
\begin{alltt}
\hlkwd{findAssocs}\hlstd{(my.dt,} \hlstr{'genomics'}\hlstd{,} \hlnum{.2}\hlstd{)}
\end{alltt}
\begin{verbatim}
$genomics
numeric(0)
\end{verbatim}
\end{kframe}
\end{knitrout}

Store words relatively to their frequency scores then finally use \verb|wordcloud| to plot the text mining results.
\begin{knitrout}
\definecolor{shadecolor}{rgb}{0.969, 0.969, 0.969}\color{fgcolor}\begin{kframe}
\begin{alltt}
\hlstd{r.matrix} \hlkwb{<-} \hlkwd{as.matrix}\hlstd{(my.dt)}
\hlstd{r.freq} \hlkwb{<-} \hlkwd{colSums}\hlstd{(r.matrix)}
\hlstd{r.freq.df} \hlkwb{<-} \hlkwd{data.frame}\hlstd{(}\hlkwc{term}\hlstd{=}\hlkwd{names}\hlstd{(r.freq),} \hlkwc{r.freq}\hlstd{=r.freq,} \hlkwc{stringsAsFactors} \hlstd{=} \hlnum{FALSE}\hlstd{)}
\hlstd{r.freq.df} \hlkwb{<-} \hlstd{r.freq.df}\hlopt\hlkwd{arrange}\hlstd{(}\hlkwd{desc}\hlstd{(r.freq))}

\hlkwd{library}\hlstd{(RColorBrewer);}
\hlstd{pal} \hlkwb{<-} \hlkwd{brewer.pal}\hlstd{(}\hlnum{5}\hlstd{,} \hlstr{"Dark2"}\hlstd{)}
\hlkwd{library}\hlstd{(wordcloud);}
\hlkwd{wordcloud}\hlstd{(r.freq.df}\hlopt{$}\hlstd{term, r.freq.df}\hlopt{$}\hlstd{r.freq,} \hlkwc{min.freq}\hlstd{=}\hlnum{20}\hlstd{,}\hlkwc{random.color}\hlstd{=}\hlnum{FALSE}\hlstd{,}\hlkwc{colors}\hlstd{=pal)}
\end{alltt}
\end{kframe}

{\centering \includegraphics[width=.49\textwidth]{graphics/plotunnamed-chunk-3} 

}



\end{knitrout}

\section{Related material}
\label{sec:material}

\noindent
\begin{itemize}
\item \href{http://www.rdatamining.com/examples/text-mining}{Text Mining}
\item \href{http://geoffjentry.hexdump.org/twitteR.pdf}{TwitteR client for R tutorial}
\item \href{http://davetang.org/muse/2013/04/06/using-the-r_twitter-package/}{Using the R twitteR package}
\item \href{http://thinktostart.com/create-twitter-sentiment-word-cloud-in-r/}{Create Twitter Sentiment Word Cloud in R}
  \item \href{http://heuristically.wordpress.com/2011/04/08/text-data-mining-twitter-r/}{Text Data Mining with Twitter and R}
\end{itemize}



\section{Session Information}
\label{sec:sess_Information}

\noindent
The version number of R and packages loaded for generating the vignette were:
\begin{knitrout}
\definecolor{shadecolor}{rgb}{0.969, 0.969, 0.969}\color{fgcolor}\begin{kframe}
\begin{verbatim}
R version 3.1.1 (2014-07-10)
Platform: x86_64-pc-linux-gnu (64-bit)

locale:
 [1] LC_CTYPE=en_CA.UTF-8      
 [2] LC_NUMERIC=C              
 [3] LC_TIME=en_CA.UTF-8       
 [4] LC_COLLATE=en_CA.UTF-8    
 [5] LC_MONETARY=en_CA.UTF-8   
 [6] LC_MESSAGES=en_CA.UTF-8   
 [7] LC_PAPER=en_CA.UTF-8      
 [8] LC_NAME=C                 
 [9] LC_ADDRESS=C              
[10] LC_TELEPHONE=C            
[11] LC_MEASUREMENT=en_CA.UTF-8
[12] LC_IDENTIFICATION=C       

attached base packages:
[1] stats     graphics  grDevices utils    
[5] datasets  methods   base     

other attached packages:
 [1] wordcloud_2.5      RColorBrewer_1.0-5
 [3] tm_0.6             NLP_0.1-5         
 [5] plyr_1.8.1         twitteR_1.1.8     
 [7] dplyr_0.2          ROAuth_0.9.3      
 [9] digest_0.6.4       RCurl_1.95-4.3    
[11] bitops_1.0-6       knitr_1.6         

loaded via a namespace (and not attached):
 [1] assertthat_0.1  bit_1.1-12     
 [3] bit64_0.9-4     codetools_0.2-9
 [5] compiler_3.1.1  evaluate_0.5.5 
 [7] formatR_1.0     highr_0.3      
 [9] httr_0.5        magrittr_1.0.1 
[11] parallel_3.1.1  Rcpp_0.11.2    
[13] rjson_0.2.14    slam_0.1-32    
[15] stringr_0.6.2   tools_3.1.1    
\end{verbatim}
\end{kframe}
\end{knitrout}


\end{linenumbers}
\end{document}










